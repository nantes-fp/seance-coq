\documentclass[9pt]{beamer}
\usepackage[french]{babel}
\usepackage[utf8]{inputenc}

\usepackage{verbatim}
\usepackage{listings}
\lstset{
  upquote=true,
  columns=flexible,
  basicstyle=\ttfamily,
  keywordstyle=\color{blue},
  commentstyle=\color{gray},
  morekeywords={Theorem,Definition,Proof,Qed,Fixpoint,Inductive,Example}
}

\usetheme{Warsaw}
\setbeamertemplate{navigation symbols}{} % Supprimer symboles de navigation

\title{Séminaire Coq}
\subtitle{Introduction et utilisation basique}
\author{Maxime Folschette}
\date{14 février 2013}

\newcommand{\todo}[1]{\textcolor{red}{[[#1]]}}

\newenvironment{ex}[1][Exemple]{\begin{exampleblock}{#1}}{\end{exampleblock}}
\newenvironment{warn}[1][Attention]{\begin{alertblock}{#1}}{\end{alertblock}}
\newenvironment{code}[1][Code]{\begin{block}{#1}}{\end{block}}

\newcommand{\coq}[1]{\texttt{\textcolor{blue}{#1}}}
\newcommand{\defi}[1]{\textbf{#1}}

\newcommand{\bs}{\symbol{92}}
\newcommand{\ctpinform}{6}
\newcommand{\ctpform}{3}
\newcommand{\N}{\mathbb{N}}
\newcommand{\R}{\mathbb{R}}



\begin{document}

\begin{frame}[plain]
\titlepage
\end{frame}



\begin{frame}[containsverbatim]
\frametitle{Définitions}

\begin{code}[Définition inductive $\rightsquigarrow$ Nouveaux types]
\begin{lstlisting}
Inductive bool : Type :=
  | true : bool
  | false : bool.
\end{lstlisting}
\end{code}

\begin{code}[Définition avec pattern matching $\rightsquigarrow$ Déconstruction]
\begin{lstlisting}
Definition negb (b:bool) : bool :=
  match b with
    | true => false
    | false => true
  end.
\end{lstlisting}
\end{code}

\end{frame}



\begin{frame}[containsverbatim]
\frametitle{Théorèmes et démonstrations}

\begin{code}[Définition d'un théorème et écriture de sa preuve]
\begin{lstlisting}
Theorem <nom> : <proposition>.
Proof.
  <tactiques et conclusions>
Qed.
\end{lstlisting}
\end{code}

\begin{code}[Environnement de preuves]
\begin{lstlisting}[]
  Contexte (variables et hypotheses)
  ==================
    Objectif actif

Objectifs en attente
...
\end{lstlisting}
\end{code}

$\Rightarrow$ On cherche à prouver l'objectif actif avant de passer aux autres.

\end{frame}



\begin{frame}[c]
\frametitle{Propositions et preuves}

\defi{Proposition} = Assertion, « phrase » non simplifiable.
\begin{ex}
\begin{itemize}
  \item $2 < 5$
  \item $\forall x \in \R, x^2 \geq 0$
  \item $\forall x \in \R, \exists y_1, y_2 \in \R, y_1 \not= y_2 \wedge y_1^2 = y_2^2 = x$
  \item $5 < 2$
\end{itemize}
\end{ex}

\pause
\bigskip
\defi{Preuve} = Succession de tactiques pour résoudre chaque objectif.
\begin{ex}
\begin{itemize}
  \item \coq{simpl} $\rightarrow$ Simplifier l'objectif courant.
  \item \coq{reflexivity} $\rightarrow$ Résoudre l'objectif courant s'il a la forme : $X = X$.
  \item \coq{intros}\texttt{ a} $\rightarrow$ On pose $a$ (pour répondre à $\forall a, ...$).
  \item \coq{intros}\texttt{ H} $\rightarrow$ On suppose $H$ (pour répondre à $H \Rightarrow ...$).
  \item \coq{rewrite}\texttt{ H} $\rightarrow$ Si $H: x = y$, remplacer tout $x$ par $y$.
  \item \coq{apply}\texttt{ H} $\rightarrow$ Conclure en appliquant $H$.
\end{itemize}
\end{ex}

\end{frame}



\begin{frame}
\frametitle{Définitions récursives}

\defi{Type paramétré} = Type dépendant d'un autre type.
\begin{ex}[\texttt{option X} --- Une option sur le type \texttt{X} est :]
\begin{itemize}
  \item soit un élément qui ne contient aucune valeur $\rightsquigarrow$ \texttt{None}
  \item soit un élément qui contient une valeur de type \texttt{X} $\rightsquigarrow$ \texttt{Some x}
\end{itemize}
\end{ex}

\pause
\medskip
\defi{Type récursif} = Type qui fait référence à lui-même.
\begin{ex}[\texttt{nat} --- Un entier naturel \texttt{nat} (dans l'arithmétique de Peano) est :]
\begin{itemize}
  \item soit l'élément nul $\rightsquigarrow$ \texttt{O}
  \item soit le successeur d'un entier naturel $\rightsquigarrow$ \texttt{S n}
\end{itemize}
\end{ex}

\pause
\medskip
Exemple de type récursif paramétré : les listes.
\begin{ex}[\texttt{list X} --- Une liste d'éléments de type \texttt{X} est :]
\begin{itemize}
  \item soit la liste vide $\rightsquigarrow$ \texttt{[]}
  \item soit un élément de \texttt{X} (tête) et une liste de \texttt{X} (queue) $\rightsquigarrow$ \texttt{h :: t}
\end{itemize}
\end{ex}
\end{frame}



\begin{frame}
\frametitle{Fonctions récursives}

\defi{Fonction récursive} = Fonction qui fait référence à elle-même.

\medskip
$\rightarrow$ Problème : quid des fonctions récursives qui ne terminent pas ?\\
Quel résultat ? Quel type ?
\begin{warn}[]
\centering$\Rightarrow$ Mathématiquement non défini
\end{warn}

\pause
\medskip
$\rightarrow$ Solution : forcer la terminaison\\
\defi{Décroissance structurelle} = Une fonction ne peut s'appeler elle-même qu'avec au moins un argument strictement inférieur structurellement\\
\defi{$X$ est structurellement inférieur à $Y$} = On peut construire $Y$ à partir de $X$\\
\begin{ex}[]
\centering$\Rightarrow$ On finit toujours dans un cas dégénéré
\end{ex}

\pause
\begin{ex}
La fonction \texttt{length} s'appelle elle-même sur la queue de la liste, qui est structurellement plus petite. (On peut construire la liste de départ à partir de sa tête et de sa queue.)
\end{ex}

\end{frame}



\begin{frame}
\frametitle{Preuves par récurrence}

\begin{ex}[Récurrence sur $\N$ (scolaire)]
Si on parvient à montrer :
\begin{itemize}
 \item $P_0$,
 \item $\forall n \in \N, P_n \Rightarrow P_{n+1}$,
\end{itemize}
Alors on a :
\begin{itemize}
 \item $\forall n \in \N, P_n$.
\end{itemize}
Autrement dit : 
$(P_0 \wedge \forall n \in \N, P_n \Rightarrow P_{n+1}) \Rightarrow \forall n \in \N, P_n$.
\end{ex}

\pause
\begin{code}[Récurrence sur “\texttt{nat}” (structurelle)]
$P(\texttt{O}) \Rightarrow
  (\forall \texttt{n} \in \texttt{nat}, P(\texttt{n}) \Rightarrow P(\texttt{S n})) \Rightarrow
  \forall \texttt{n} \in \texttt{nat}, P(\texttt{n})$
\end{code}

\begin{code}[Récurrence sur “\texttt{list X}” (structurelle)]<3>
$P(\texttt{[]}) \Rightarrow
  (\forall \texttt{x} \in \texttt{X}, \forall \texttt{l} \in \texttt{list X}, P(\texttt{l}) \Rightarrow P(\texttt{x :: l})) \Rightarrow
  \forall \texttt{l} \in \texttt{list X}, P(\texttt{l})$
\end{code}

avec associativité à droite de “$\Rightarrow$” :
\\\quad$A \Rightarrow B \Rightarrow C$
\quad$\equiv\quad A \Rightarrow (B \Rightarrow C)$
\quad$\equiv\quad (A \wedge B) \Rightarrow C$
\end{frame}



\begin{frame}
\frametitle{Un petit mot sur Curry-Howard}

\huge

\begin{center}
\texttt{H : P -> Q}

\pause
\texttt{f : P -> Q}
\end{center}

\end{frame}



\begin{frame}
\institute{\url{https://github.com/nantes-fp/seance-coq}}
\frametitle{Références et Licence}
\maketitle

\url{http://coq.inria.fr/}

\url{http://www.cis.upenn.edu/~bcpierce/sf/}

\url{http://www.coursera.org/course/progfun}

\url{http://www.labri.fr/perso/casteran/CoqArt/index.html}

\rule{5cm}{.1pt}

Licence Beerware / CC0 : si les termes de la licence Beerware ne peuvent s'appliquer ou semblent trop restrictifs, ceux de la licence CC0 s'appliquent.
Réutilisation encouragée.

\end{frame}

\end{document}
