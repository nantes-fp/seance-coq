\documentclass[9pt]{beamer}
\usepackage[french]{babel}
\usepackage[utf8]{inputenc}

\usepackage{verbatim}
\usepackage{listings}
\lstset{
  upquote=true,
  columns=flexible,
  basicstyle=\ttfamily,
  keywordstyle=\color{blue},
  commentstyle=\color{gray},
  morekeywords={Theorem,Definition,Proof,Qed,Fixpoint,Inductive,Example}
}

\usetheme{Warsaw}
\setbeamertemplate{navigation symbols}{} % Supprimer symboles de navigation

\title{Séminaire Coq}
\subtitle{Introduction et utilisation basique}
\author{Maxime Folschette}
\date{27 novembre 2012}

\newcommand{\todo}[1]{\textcolor{red}{[[#1]]}}

\newenvironment{ex}[1][Exemple]{\begin{exampleblock}{#1}}{\end{exampleblock}}
\newenvironment{warn}[1][Attention]{\begin{alertblock}{#1}}{\end{alertblock}}
\newenvironment{code}[1][Code]{\begin{block}{#1}}{\end{block}}

\newcommand{\coq}[1]{\texttt{\textcolor{blue}{#1}}}
\newcommand{\defi}[1]{\textbf{#1}}



\begin{document}

\begin{frame}[plain]
\institute{\url{http://www.irccyn.ec-nantes.fr/~folschet/coq/exos1-vide.v}}
\titlepage
\institute{}
\end{frame}

\begin{frame}
\frametitle{Coq = assistant de \textcolor{red}{preuves} formelles}

\defi{Preuve} = Démonstration\\
~\phantom{\defi{Preuve}}= Raisonnement propre à établir une vérité\\
~\phantom{\defi{Preuve}= }à partir de propositions initiales
\begin{ex}
  Les chats ont des poils et Socrate est un chat, donc Socrate a des poils.
\end{ex}

\pause
\only<2>{
\begin{ex}
Tous ceux qui écoutent du Black Metal ont les cheveux longs.\\J'ai les cheveux longs donc j'écoute du Black Metal.
\end{ex}
}

\pause
\only<1,3->{
\begin{warn}[Exemple de raisonnement faux]
Tous ceux qui écoutent du Black Metal ont les cheveux longs.\\J'ai les cheveux longs donc j'écoute du Black Metal.
\end{warn}
}

\begin{ex}
Je suis à Londres $\Rightarrow$ Je suis en Angleterre\\\quad
$\nLeftrightarrow$ Je ne suis pas à Londres $\Rightarrow$ Je ne suis pas en Angleterre
\end{ex}

Démonstration par récurrence, par l'absurde, ...

\end{frame}



\newcommand{\ctpinform}{6}
\newcommand{\ctpform}{3}
\newcommand{\N}{\mathbb{N}}

\begin{frame}[t]
\frametitle{Coq = assistant de preuves \textcolor{red}{formelles}}

Distinction formel / informel

\pause
\defi{Preuve informelle} = Lisible par un être humain
\only<-\ctpinform>{
\begin{ex}[MQ : Il y a autant d'entiers naturels pairs que d'entiers naturels impairs.]
\begin{enumerate}
\pause
  \item À chaque entier naturel pair, on peut associer son successeur qui est impair. \uncover<\ctpinform->{\textcolor{red}{(Pourquoi ?)}}\\
        De même, à chaque entier naturel impair, on peut associer son prédécesseur qui est pair. \uncover<\ctpinform->{\textcolor{red}{(Pourquoi ?)}}
\pause
  \item À chaque entier naturel pair, on peut associer son successeur qui est impair. \uncover<\ctpinform->{\textcolor{red}{(Pourquoi ?)}}\\
        De même pour tout entier naturel impair. \uncover<\ctpinform->{\textcolor{red}{(Vraiment ?)}}
\pause
  \item C'est trivial. \uncover<\ctpinform->{\textcolor{red}{(Alors montrez-le.)}}
\end{enumerate}
\end{ex}
}

\pause[\ctpinform]
\pause
\defi{Preuve formelle} = Objet mathématique, lisible par une machine 
\begin{ex}[MQ : Il y a autant d'entiers naturels pairs que d'entiers naturels impairs.]
\pause
{}[ MQ : $\exists f : P \rightarrow Q \text{ bijective, avec : }$\\$P = \{n \in \N \mid \exists k \in \N, n = 2k\} \wedge Q = \{m \in \N \mid \exists k \in \N, m = 2k+1\}$ ]

\uncover<11->{
Soit $n \in P$. Alors $\exists k \in \N, n = 2k$, ainsi $n+1 = 2k+1 \in Q$.

Soit $m \in Q$. Alors $\exists l \in \N, m = 2l+1$, ainsi $m-1 = 2l \in P$.
}

\uncover<12->{
[ Montrer aussi que $m \geq 1$ sans quoi on pourrait avoir $m-1 < 0$. ]

[ Pour cela, montrer que $Q$ est minoré et trouver son minimum. ]
}

\pause
Soient :
$f = \left\{\begin{array}{ccc}
  P & \rightarrow & Q \\
  n & \rightarrow & n + 1
\end{array}\right.$
et
$g = \left\{\begin{array}{ccc}
  Q & \rightarrow & P \\
  m & \rightarrow & m - 1
\end{array}\right.$

\pause
Soit $m \in Q$. On a : $f \circ g (m) = f(g(m)) = f(m - 1) = (m - 1) + 1 = m$.

Soit $n \in P$. On a : $g \circ f (n) = g(f(n)) = g(n + 1) = (n + 1) - 1 = n$.

Ainsi, $f$ est une bijection de $P$ dans $Q$ (de réciproque $g$). CQFD.
\end{ex}
\end{frame}



\begin{frame}
\frametitle{Coq = \textcolor{red}{assistant} de preuves formelles}

\defi{Assistant} = Coq ne crée pas de démonstration, il se contente de vérifier celle que vous écrivez.

\bigskip
Coq est le correcteur de khôle qui hurle dès que vous essayez de l'embrouiller.
\end{frame}



\begin{frame}
\frametitle{Applications de Coq}

\begin{itemize}
 \item Démonstration de théorèmes complexes
\begin{itemize}
  \item c'est rare car la formulation est difficile
  \item mais de plus en plus fréquent pour s'assurer de la véracité d'une preuve
  \item et c'est enrichissant pour la communauté scientifique
\end{itemize}

\begin{ex}
Théorème des quatre couleurs
\end{ex}

\bigskip
\pause
 \item Preuve de programmes
\begin{itemize}
  \item s'assurer qu'un programme possède le comportement voulu
  \item s'assurer qu'un comportement indésirable n'est pas possible
  \item de plus en plus fréquent
\end{itemize}
\begin{ex}
Électronique embarquée et critique (voiture, avion, navettes…)
\end{ex}

\end{itemize}

\end{frame}



\begin{frame}
\frametitle{Gallina}

Le langage de définitions de Coq est \textbf{Gallina} (inspiré d'OCaml). Il permet de définir :
\begin{itemize}
  \item des fonctions $\rightarrow$ sur lesquelles porteront les preuves,
  \item des propositions $\rightarrow$ qui pourront servir d'énoncés de théorèmes,
  \item de lancer une démonstration...
\end{itemize}

\pause
\medskip
C'est un langage purement fonctionnel
\begin{itemize}
  \item Vraiment purement fonctionnel
  \item Transparence référentielle (pas d'exception, pas d'E/S...)
  \item[$\Rightarrow$] Permet d'écrire des maths
\end{itemize}

\end{frame}



\newcommand{\bs}{\symbol{92}}
\begin{frame}
\frametitle{Les propositions}

Les propositions permettent d'écrire des assertions mathématiques.

\medskip
Les symboles :
\begin{itemize}
  \item \texttt{/\bs}, \texttt{\bs/} : conjonction, disjonction
  \item \texttt{->}, \texttt{<->} : implication, équivalence
  \item \texttt{=}, \texttt{<>} : égalité, inégalité
  \item \texttt{<}, \texttt{<=}, \texttt{>}, \texttt{>=}, : comparaisons (entiers naturels seulement)
\end{itemize}

\medskip
Contrairement aux fonctions, elles sont figées (non simplifiables).

\pause
\begin{warn}[]
\textbf{Elles peuvent représenter des assertions fausses !} Une proposition, contrairement à un théorème, n'est pas suivie d'une démonstration et peut énoncer n'importe quoi de syntaxiquement correct.
\end{warn}

\end{frame}



\begin{frame}[containsverbatim]
\frametitle{Les théorèmes et démonstrations}

\begin{columns}[t]
  \begin{column}{5cm}
    \begin{code}[Définition d'un théorème]
    \begin{lstlisting}
Theorem <nom> : <proposition>.
Proof.
  <tactiques et conclusions>
Qed.
    \end{lstlisting}
    \end{code}
  \end{column}
  \begin{column}{5cm}
    \begin{code}[Environnement de preuves]
    \begin{lstlisting}
  Contexte
  ==================
   Objectif actif

Objectifs en attente
    \end{lstlisting}
    \end{code}
  \end{column}
\end{columns} 

\begin{itemize}
  \item Définir un théorème lance l'environnement de preuves.
  \item L'environnement de preuves contient :
  \begin{itemize}
   \item un ensemble d'objectifs (propositions à prouver pour résoudre la démonstration) --- un seul objectif est actif à la fois,
   \item un contexte (hypothèses).
  \end{itemize}
  \item Au départ, le seul objectif est l'énoncé du théorème et le contexte est vide.
  \item Les tactiques permettent de résoudre la démonstration. Elles peuvent avoir trois effets :
  \begin{itemize}
    \item modifier l'objectif courant,
    \item créer un nouvel objectif (rajouter une étape dans la démonstration),
    \item supprimer l'objectif courant (résoudre l'étape en cours).
  \end{itemize}
  \item On peut clore une démonstration (\coq{Qed}\texttt{.}) lorsqu'il ne reste plus d'objectif.
\end{itemize}

\end{frame}



\begin{frame}
\frametitle{Définitions récursives}

\defi{Type paramétré} = Type dépendant d'un autre type.
\begin{ex}[Une \texttt{option} sur le type \texttt{X} est :]
\begin{itemize}
  \item soit un élément qui ne contient aucune valeur $\rightarrow$ \texttt{None}
  \item soit un élément qui contient une valeur de type \texttt{X} $\rightarrow$ \texttt{Some x}
\end{itemize}
\end{ex}

\pause
\defi{Type récursif} = Type qui fait référence à lui-même.
\begin{ex}[Un entier naturel \texttt{nat} (arithmétique de Peano) est :]
\begin{itemize}
  \item soit l'élément nul $\rightarrow$ \texttt{O}
  \item soit le successeur d'un entier naturel $\rightarrow$ \texttt{S n}
\end{itemize}
\end{ex}

\pause
Les listes :
\begin{ex}[Une liste d'éléments de type \texttt{X} (\texttt{list X}) est :]
\begin{itemize}
  \item soit la liste vide $\rightarrow$ \texttt{[]}
  \item soit un élément de \texttt{X} (tête) et une liste de \texttt{X} (queue) $\rightarrow$ \texttt{h :: t}
\end{itemize}
\end{ex}
\end{frame}



\begin{frame}
\frametitle{Fonctions récursives}

\defi{Fonction récursive} = Fonction qui fait référence à elle-même.

\medskip
$\rightarrow$ Problème : quid des fonctions récursives qui ne terminent pas ?\\
Quel résultat, Quel type ?\\
\quad$\Rightarrow$ Mathématiquement non défini

\pause
\medskip
$\rightarrow$ Solution : forcer la terminaison\\
\defi{Décroissance structurelle} = Une fonction ne peut s'appeler elle-même qu'avec au moins un argument strictement inférieur structurellement\\
\defi{$X$ est structurellement inférieur à $Y$} = On peut construire $Y$ à partir de $X$\\
\quad$\Rightarrow$ On finit toujours dans un cas dégénéré

\pause
\begin{ex}
La fonction \texttt{length} s'appelle elle-même sur la queue de la liste, qui est structurellement plus petite. (On peut construire la liste de départ à partir de sa tête et de sa queue.)
\end{ex}


\end{frame}



\begin{frame}
\frametitle{Preuves par récurrence}

\begin{ex}[Récurrence sur $\N$ (scolaire)]
Si on parvient à montrer $P_0$ et $\forall n \in \N, P_n \Rightarrow P_{n+1}$, alors on a : $\forall n \in \N, P_n$.
\\Autrement dit : 
$(P_0 \wedge \forall n \in \N, P_n \Rightarrow P_{n+1}) \Rightarrow \forall n \in \N, P_n$.
\end{ex}

\pause
\begin{code}[Récurrence sur “\texttt{nat}” (structurelle)]
$P(\texttt{O}) \Rightarrow
  (\forall \texttt{n} \in \texttt{nat}, P(\texttt{n}) \Rightarrow P(\texttt{S n})) \Rightarrow
  \forall \texttt{n} \in \texttt{nat}, P(\texttt{n})$
\end{code}

\begin{code}[Récurrence sur “\texttt{list X}” (structurelle)]<3>
$P(\texttt{[]}) \Rightarrow
  (\forall \texttt{x} \in \texttt{X}, \forall \texttt{l} \in \texttt{list X}, P(\texttt{l}) \Rightarrow P(\texttt{x :: l})) \Rightarrow
  \forall \texttt{l} \in \texttt{list X}, P(\texttt{l})$
\end{code}

avec associativité à droite de “$\Rightarrow$” :
\\\quad$A \Rightarrow B \Rightarrow C$
\quad$\equiv\quad A \Rightarrow (B \Rightarrow C)$
\quad$\equiv\quad (A \wedge B) \Rightarrow C$
\end{frame}



\begin{frame}
\frametitle{Un petit mot sur Curry-Howard}

\huge

\begin{center}
\texttt{H : P -> Q}

\pause
\texttt{f : P -> Q}
\end{center}

\end{frame}



\begin{frame}
\institute{\url{http://www.irccyn.ec-nantes.fr/~folschet/coq/}}
\frametitle{Références et Licence}
\maketitle

\url{http://coq.inria.fr/}

\url{http://www.cis.upenn.edu/~bcpierce/sf/}

\url{http://www.coursera.org/course/progfun}

\url{http://www.labri.fr/perso/casteran/CoqArt/index.html}

\medskip
Licence : Beerware, réutilisation encouragée

\end{frame}

\end{document}
